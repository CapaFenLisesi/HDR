% #####################################################
% #####################################################
% #####################################################
\setchapterpreamble[u]{\margintoc}
\chapter{Multipactor}\label{chap:Multipactor}



Since RF breakdowns are a recurrent problematic in high RF power applications, the last section of this chapter~\ref{sec:Multipactor} is fully dedicated to this topic, in particular to the multipactor phenomenon which occurs in vacuum environments and led to collaboration between CNES, ONERA and CEA since 2014.

reviews:\cite{kishek1998}

Models\cite{vaughan1989}

Multipacting is a Resonant Electron Dischargewith Electron Multplication.•To have multipacting you needthe occurrence oftwo conditions1)electron synchronizationwith the RF Field.2)Electron multiplication via Secondary electronreemission.

Resonant RF vacuum discharge sustained by secondary electron emission•One or two surfaces involved•Multiple modes•Signs of multipactor:–Heating–Changed r.f. performance–Window failure–Light and X-ray emission•Multipactoron dielectric surfaces does not require RF field•Multipactorcan sometimes be suppressed by–Changing shape of surface–Surface coatings–Static electric and magnetic fields

Dégradations des passages étanches des antennes FCI (origine?)
\sidecite{vulliez2007}



LH Windows Multipactor
\sidecite{kim2007}


\marginnote{Part of this section are taken from \citeauthyear{hillairet2017}, PhDs \citeauthyear{fil2017-1, placais2020} and associated papers \cite{fil2016, fil2017, fil2017-2, fil2018, placais2018, placais2019}.}

Magnetic Fusion RF antennas are generally made of copper or silver-coated stainless steels located in vacuum environments. The vacuum sealing with pressurized transmission lines is made with the help of ceramics such as alumina, aluminium nitride, beryllium oxide or diamond. 

All these components are subject to multipactor discharges which increases the electron population (via secondary emission and gas desorption) which in turn may ultimately lead to an avalanche effect and the development of a discharge even at pressures two order of magnitude below Paschen breakdown limit\sidecite{graves2006}. These discharges are generally considered detrimental since they can lead to detuned RF systems, limit the RF power transmission in the plasma and eventually damage RF sources or components. When not detected quickly enough, arcs can lead to surface erosion \sidecite[-0.4cm]{goniche2012-2}, dielectric components metallisation \sidecite{wang2015} or water/air leaks due to punctured components such as bellows \sidecite{mayoral2007} or vacuum windows \sidecite{neuber1998, neuber2007, hillairet2015}. 

In some cases however, multipactor-induced discharges can be desired for vacuum RF conditioning during short RF pulses and at moderate power\sidecite{goniche2012-2, wang2015, halbritter1982, ekedahl1998}. Moreover, these RF systems are subjected to the high magnetic field environment of the experiments in the Tesla range. The presence of magnetic field affects the electron trajectories and thus the multipactor resonances. On tokamaks, the electron motion is strongly constrained around the magnetic field lines which reduce the effects of loss mechanisms such as diffusion and increases electron impact ionization and thus the build-up of glow discharges. Finally, at the difference of RF payloads in satellite, the antenna surfaces can be polluted during operation with particles resulting from the strong interaction of the energetic particles with the walls of the tokamak, which may alter the surface characteristics such as secondary emission. This paper reviews the work performed in the fusion research community on multipactor discharges for two kinds of RF systems with their practical implications on power delivery into the plasma. The Ion Cyclotron Resonance Heating (ICRH), which uses coaxial lines in the MHz range of frequency, is presented in the next section. The Lower Hybrid Current Drive (LHCD) systems, which uses rectangular waveguides in the GHz range of frequency is discussed in section 3. 

\todo{Power limit and multipactor in grill:\cite{hwang1981, vaughan1982, goniche2012-2, goniche2014} from vaughan1982:
	This suggests that multipactor in the Brambilla grill might be suppressed
	by making each of the dividing septa thicker in the middle than at the
	edges, to introduce a similar small lateral component of RF field. If
	successful, this would have the merit of being a "geometrical" not
	subject to change during the life of the equipment. Solutions to multipactor
	which depend on surface treatment orc onditioning are inherently
	less reliable, because further surface changes during life are not only
	possible but very probable.
	If tests are made of this suggestion, it is essential that the actual working
	environment be reproduced, most particularly the local magnetic
	field. Multipactor is a phenomenon strongly influenced by stray fields,
	and testsi n which they are norte presented are almost valueless.
	
	
	
	use of titatium to lower secondary emission in JFT-2 and WEGA
	not a great idea in CMod
}