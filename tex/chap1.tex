% controlled fusion and RF heating and current drive
%\chaptertoc{}

\chapter{Controlled Fusion, RF heating and Current Drive}

%%%%%%%%%%%%%%%%%%%%%%%%%%%%%%%%%%%%%%%%%%
%%%%%%%%%%%%%%%%%%%%%%%%%%%%%%%%%%%%%%%%%%
\section{Controlled Fusion}


%%%%%%%%%%%%%%%%%%%%%%%%%%%%%%%%%%%%%%%%%%

\subsection{Magnetic Confinement}
%%%%%%%%%%%%%%%%%%%%%%%%%%%%%%%%%%%%%%%%%%

\subsection{Ohmic Heating}
%%%%%%%%%%%%%%%%%%%%%%%%%%%%%%%%%%%%%%%%%%

\subsection{Radio frequency Heating and Current Drive}
%%%%%%%%%%%%%%%%%%%%%%%%%%%%%%%%%%%%%%%%%%
\section{Ion Cyclotron Resonance Frequency}

%%%%%%%%%%%%%%%%%%%%%%%%%%%%%%%%%%%%%%%%%%
%%%%%%%%%%%%%%%%%%%%%%%%%%%%%%%%%%%%%%%%%%
\section{Lower Hybrid Resonance Frequency}

%%%%%%%%%%%%%%%%%%%%%%%%%%%%%%%%%%%%%%%%%%
Originally, the occurrence of a wave resonance, the \emph{lower hybrid resonance}, has been anticipated to lead to strong wave-particle interaction through linear and non-linear mode conversion to a hot plasma wave\sidecite{Stix1992}. With an appropriate RF launcher conceived to excite cold plasma waves, these would propagate into the plasma until reaching the lower hybrid resonant layer at $\omega_{LH}$. This resonance exists in tokamak plasma in the region close to the ion plasma frequency $\omega_{pi}/2\pi$, which lies in the lower end of the microwave band (1-5~GHz). At this layer, the perpendicular group velocity vanishes and the waves can convert into a hot plasma mode which is absorbed. This heating technique, known as \emph{Lower Hybrid plasma Ion Heating} (LHIH) or \emph{Lower Hybrid Resonance Heating} (LHRH), was the originally experimentally investigated method in the 70'\sidecite{Bellan1974, Hooke1972, Golant1972, Tonon1977}. Different physical mechanisms have been invoked to explain the energy absorption, such as stochastic Ion Heating in \citeauthyear{Karney1978a} and quasi-linear electron Landau damping in \citeauthyear{Brambilla1983}.


In the 80', effective ion heating had only been obtained in a small number of experiments and research along the application of LH waves towards bulk ion heating were slowing down\sidecite{Gormezano1986, Porkolab1984a, Tonon1984}. The reason for this is that bulk ion heating near the mode conversion layer appeared to be less reproducible and more difficult to achieve than electron heating. Indeed, as the wave frequency gets closer to the lower hybrid frequency, the shorter wavelength waves may be more effectively absorbed and/or scattered near the plasma surface by non-linear effects such as parametric instabilities, low-frequency fluctuations, etc. Moreover, for LH bulk ion heating, the unconfined ions impinging on the wall induced a large amount of metallic impurities and then the increase of power radiated by the plasma.

Rather than trying to heat ions, it was theorized postulated that high phase velocity waves travelling in the direction parallel to the magnetic field could interact quasi-linearly by Landau interaction with the electrons population, and, by using an asymmetric spectrum could drive a large amount additional of toroidal plasma current \sidecite{Fisch1978}. In the same fashion that for LHRH, the RF power is coupled to the plasma via launchers made of rectangular waveguides stacked periodically in the horizontal direction parallel to the toroïdal magnetic field. However, at the contrary of LHRH launchers, the LH waves are launched preferentially in one toroidal direction by mean of a phased array. The LH wave excited by such an array has an asymmetric parallel spectrum. The LH waves create an asymmetry in the electron distribution, which ultimately results in a net electric current \sidecite{Fisch1987}. This technique is known as \emph{Lower Hybrid Current Drive} and despite the fact that the Lower Hybrid resonance is not any more involved in the use of this method in tokamaks, the term remained. LHCD has been confirmed on the PLT tokamak in 1982 \sidecite{Bernabei1982, Motley1985, Jobes1985} and in Alcator C in 1984 \sidecite{Porkolab1984}. 

Since in 1982 many impressive results were presented on LHCD\sidecite{Hooke1982, Porkolab1984, Tonon1982} toward steady state or quasi steady state tokamak operations, most LH experiments were dedicated to electron interaction and especially to current drive. A recent review of LHCD is available in \sidecite{Bonoli2014}.

Currently, the Lower Hybrid waves term refers to the waves which satisfy the slow-wave branch of the cold plasma dispersion relation for parallel index larger than one ($|n_{\parallel}|>1$) and a RF frequency $\omega$ which lies between the ion cyclotron $\omega_{ci}$ and the electron cyclotron $\omega_{ce}$ frequencies. 

For the LH (lower hybrid) method which operates at the lower end of the microwave band (1-5 GHz) klystrons transform electrical power into electromagnetic power (step 1), which is transported to the plasma using waveguides (step 2). The power is coupled to the plasma with antennas called "grills" because of their characteristic shape (step 3), transported inside the plasma by plasma waves (typically the slow wave) (step 4), and absorbed on ions or electrons by wave-particle interaction (step 5).



